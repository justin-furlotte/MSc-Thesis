%%
%% This is file `ubcsample.tex',
%% generated with the docstrip utility.
%
% The original source files were:
%
% ubcthesis.dtx  (with options: `ubcsampletex')
%% 
%% This file was generated from the ubcthesis package.
%% --------------------------------------------------------------
%% 
%% Copyright (C) 2001
%% Michael McNeil Forbes
%% mforbes@alum.mit.edu
%% 
%% This file may be distributed and/or modified under the
%% conditions of the LaTeX Project Public License, either version 1.2
%% of this license or (at your option) any later version.
%% The latest version of this license is in
%%    http://www.latex-project.org/lppl.txt
%% and version 1.2 or later is part of all distributions of LaTeX
%% version 1999/12/01 or later.
%% 
%% This program is distributed in the hope that it will be useful,
%% but WITHOUT ANY WARRANTY; without even the implied warranty of
%% MERCHANTABILITY or FITNESS FOR A PARTICULAR PURPOSE.  See the
%% LaTeX Project Public License for more details.
%% 
%% This program consists of the files ubcthesis.dtx, ubcthesis.ins, and
%% the sample figures fig.eps and fig.fig.
%% 
%% This file may be modified and used as a base for your thesis without
%% including the licence agreement as long as the content (i.e. textual
%% body) of the file is completely rewritten. You must, however, change
%% the name of the file.
%% 
%% This file may only be distributed together with a copy of this
%% program. You may, however, distribute this program without generated
%% files such as this one.
%% 

% This Sample thesis requires \LaTeX2e
\NeedsTeXFormat{LaTeX2e}[1995/12/01]
\ProvidesFile{ubcsample.tex}[2015/05/31 v1.72 ^^J
 University of British Columbia Sample Thesis]
% This is the \documentclass[]{} command.  The manditory argument
% specifies the "flavour" of thesis (ubcthesis for UBC).  The
% optional arguments (in []) specify options that affect how the
% thesis is displayed.  Please see the ubcthesis documentation for
% details about the options.
\documentclass[msc,oneside]{ubcthesis}
%
% To compile this sample thesis, issue the following commands:
% latex ubcsample
% bibtex ubcsample
% latex ubcsample
% latex ubcsample
% latex ubcsample
%
% To view use xdvi (on unix systems):
% xdvi ubcsample.dvi
%
% To make a postscript file, use dvips:
% dvips -o ubcsample.ps ubcsample.dvi
%
% To view the postscript file, use ghostview or gv (on unix systems):
% gv ubcsample.ps
%
%************************************************
% Optional packages.
%
% The use of these packages is optional, but they provide various
% tools for more flexible formating.  The sample thesis uses these,
% but if you remove the example code, you should be able to exclude
% these packages.  Only standard packages have been described here;
% they should be installed with any complete LaTeX instalation, but
% if not, you can find them at the Comprehensive TeX Archive Network
% (CTAN): http://www.ctan.org/
%

%******** afterpage ***************************
% This package allows you to issue commands at the end of the current
% page.  A good use for this is to use the command
% \afterpage{\clearpage} right after a figure.  This will cause the
% figure to be inserted on the page following the current one (or on
% the current page if it will fit) but will not break the page in the
% middle.
\usepackage{afterpage}

%******** float *********************************
% This package allows you to customize the style of
% "floats"---floating objects such as figures and tables.  In
% addition, it allows you to define additional floating objects which
% may be included in a list similar to that produces by \listoftables
% and \listoffigures.  Common uses include introducing floats for
% programs and other code bits in Compute Science and Chemical Schema.
\usepackage{float}

%******** tocloft *******************************
% This package allows you to customize and define custom lists such
% as a list of programs or Chemical Scheme.  Note: if you use the
% subfigure package, you must specify that you do as an option here.
% The title option uses the default formatting.  We do not use this
% here as the default formatting is acceptable.  Use the float
% package instead unless you need the extra formatting control
% provided by tocloft.
%\usepackage[subfigure, titles]{tocloft}

%******** alltt *********************************
% The alltt package allows you to include files and have them
% formatted in a verbatim fashion.  This is useful for including
% source code from an additional file.
%\usepackage{alltt}

%******** listings ******************************
% The listings package may be used to include chunks of source code
% and has facilities for pretty-printing many languages.
%\usepackage{listings}

%******** longtable *****************************
% The longtable package allows you to define tables that span
% multiple pages.
\usepackage{longtable}

%******** graphics and graphicx *****************
% This allows you to include encapsulated postscript files.  If you
% don't have this, comment the \includegraphics{} line following the
% comment "%includegraphics" later in this file.
\usepackage{graphicx}

%******** subfigure *****************************
% The subfigure package allows you to include multiple figures and
% captions within a single figure environment.
%\usepackage{subfigure}

%******** here **********************************
% The here package gives you more control over the placement of
% figures and tables.  In particular, you can specify the placement
% "H" which means "Put the figure here" rather than [h] which means
% "I would suggest that you put the figure here if you think it looks
% good."
%\usepackage{here}

%******** pdflscape ********************************
% This allows you to include landscape layout pages by using the
% |landscape| environment.  The use of |pdflscape| is preferred over
% the standard |lscape| package because it automatically rotates the
% page in the pdf file for easier reading.  (Thanks to Joseph Shea
% for pointing this out.)
\usepackage{pdflscape}

%******** natbib ********************************
% This is a very nice package for bibliographies.  It includes options
% for sorting and compressing bibliographic entries.
\usepackage[numbers,sort&compress]{natbib}

%******** psfrag ******************************
% This allows you to replace text in postscript pictures with formated
% latex text.  This allows you to use math in graph labels
% etc. Uncomment the psfrag lines following the "%psfrag" comment
% later in this file if you don't have this package.  The replacements
% will only be visible in the final postscript file: they will be
% listed in the .dvi file but not performed.
\usepackage{psfrag}

%******** hyperref *****************************
% Please read the manual:
% http://www.tug.org/applications/hyperref/manual.html
%
% This adds hyperlinks to your document: with the right viewers (later
% versions of xdvi, acrobat with pdftex, latex2html etc.) this will
% make your equation, figure, citation references etc. hyperlinks so
% that you can click on them.  Also, your table of contents will be
% able to take you to the appropriate sections.  In the viewers that
% support this, the links often appear with an underscore.  This
% underscore will not appear in printed versions.
%
% Note: if you do not use the hypertex option, then the dvips driver
% may be loaded by default.  This will cause the entries in the list
% of figures and list of tables to be on a single line because dvips
% does not deal with hyperlinks on broken lines properly.
%
% NOTE: HYPERREF is sensitive to the ORDER in which it is LOADED.
% For example, it must be loaded AFTER natbib but BEFORE newly
% defined float environments.  See the README file with the hyperref
% for some help with this.  If you have some very obscure errors, try
% first disabling hyperref.  If that fixes the problem, try various
% orderings.
%
% Note also that there is a bug with versions before 2003/11/30
% v6.74m that cause the float package to not function correctly.
% Please ensure you have a current version of this package.  A
% warning will be issued if you leave the date below but do not have
% a current version installed.
%
% Some notes on options: depending on how you build your files, you
% may need to choose the appropriate option (such as [pdftex]) for the
% backend driver (see the hyperref manual for a complete list).  Also,
% the default here is to make links from the page numbers in the table
% of contents and lists of figures etc.  There are other options:
% excluding the [linktocpage] option will make the entire text a
% hyperref, but for some backends will prevent the text from wrapping
% which can look terrible.  There is a [breaklinks=true] option that
% will be set if the backend supports (dvipdfm for example supports
% it but does not work with psfrag.)
%
% Finally, there are many options for choosing the colours of the
% links.  These will be included by default in future versions but
% you should probably consider changing some now for the electronic
% version of your thesis.
\usepackage[unicode=true,
  linktocpage,
  linkbordercolor={0.5 0.5 1},
  citebordercolor={0.5 1 0.5},
  linkcolor=blue]{hyperref}

% If you would like to compile this sample thesis without the
% hyperref package, then you will need to comment out the previous
% \usepackage command and uncomment the following command which will
% put the URL's in a typewriter font but not link them.
%\newcommand\url[1]{\texttt{#1}}

%******** setspace *******************************
% The setspace package allows you to manually set the spacing of the
% file.  UBC may require 1.5 spacing for microfilming of theses.  In
% this case you may obtain this by including this package and issuing
% one of the following commands:
%\usepackage{setspace}
%\singlespacing
%\onehalfspacing
%\doublespacing

% These commands are optional.  The defaults are shown.  You only
% need to include them if you need a different value
\institution{The University Of British Columbia}

% If you are at the Okanagan campus, then you should specify these
% instead.
%\faculty{The College of Graduate Studies}
%\institutionaddress{Okanagan}
\faculty{The Faculty of Graduate Studies}
\institutionaddress{Vancouver}

% You can issue as many of these as you have...
\previousdegree{B.Sc., The University of British Columbia, 1999}
\previousdegree{M.Sc., The University of British Columbia, 2001}
\previousdegree{Ph.D., Massachusetts Institute of Technology, 2005}

% You can override the option setting here.
% \degreetitle{Jack of All Trades}

% These commands are required.
\title{A Sample UBC Thesis}
\subtitle{With a Subtitle}
\author{Michael M$^{\rm c}$Neil Forbes}
\copyrightyear{2000}
\submitdate{\monthname\ \number\year} % The "\ " is required after
                                      % \monthname to prevent the
                                      % command from eating the space.
\program{Physics}

% These commands are presently not required for UBC theses as the
% advisor's name and title are not presently required anywhere.
%\advisor{Ariel R.~Zhitnitsky}
%\advisortitle{Professor of Physics}

% One might want to override the format of the section and chapter
% numbers.  This shows you how to do it.  Note that the current
% format is acceptable for submission to the FoGS: If you wish to modify
% these, you should check with the FoGS explicity. prior to making
% the modifications.
\renewcommand\thepart         {\Roman{part}}
\renewcommand\thechapter      {\arabic{chapter}}
\renewcommand\thesection      {\thechapter.\arabic{section}}
\renewcommand\thesubsection   {\thesection.\arabic{subsection}}
\renewcommand\thesubsubsection{\thesubsection.\arabic{subsubsection}}
\renewcommand\theparagraph    {\thesubsubsection.\arabic{paragraph}}
\renewcommand\thesubparagraph {\theparagraph.\arabic{subparagraph}}


\graphicspath{ {c:/user/justin/grad school/Thesis} }

\setcounter{tocdepth}{2}
\setcounter{secnumdepth}{2}

% Here is an example of a "Program" environment defined with the
% "float" package.  The list of programs will be stored in the file
% ubcsample.lop and the numbering will start with the chapter
% number.  The style will be "ruled".
\floatstyle{ruled}
\newfloat{Program}{htbp}{lop}[chapter]

% Here is the start of the document.
\begin{document}

%% This starts numbering in Roman numerals as required for the thesis
%% style and is mandatory.
\frontmatter

%%% The order of the following components should be preserved.  The order
%%% listed here is the order currently required by FoGS:        \\
%%% Title (Mandatory)                                           \\
%%% Preface (Manditory if any collaborator contributions)       \\
%%% Abstract (Mandatory)                                        \\
%%% List of Contents, Tables, Figures, etc. (As appropriate)    \\
%%% Acknowledgements (Optional)                                 \\
%%% Dedication (Optional)                                       \\

\maketitle                      %% Mandatory
\begin{abstract}                %% Mandatory -  maximum 350 words
  The \texttt{genthesis.cls} \LaTeX{} class file and accompanying
  documents, such as this sample thesis, are distributed in the hope
  that it will be useful but without any warranty (without even the
  implied warranty of fitness for a particular purpose).  For a
  description of this file's purpose, and instructions on its use, see
  below.

  These files are distributed under the GPL which should be included
  here in the future.  Please let the author know of any changes or
  improvements that should be made.

  Michael Forbes.
  mforbes@physics.ubc.ca
\end{abstract}

\chapter{Preface} % Manditory if any of the conditions are met

You must include a preface if any part of your research was partly or
wholly published in articles, was part of a collaboration, or required
the approval of UBC Research Ethics Boards.

The Preface must include the following:

\begin{itemize}
\item A statement indicating the relative contributions of all
  collaborators and co-authors of publications (if any), emphasizing
  details of your contribution, and stating the proportion of research
  and writing conducted by you.
\item A list of any publications arising from work presented in the
  dissertation, and the chapter(s) in which the work is located.
\item The name of the particular UBC Research Ethics Board, and the
  Certificate Number(s) of the Ethics Certificate(s) obtained, if
  ethics approval was required for the research.
\end{itemize}

%%% Sections and subsections etc. in the Preface should in general
%%% not be listed in the table of contents, so use the starred form
%%% of \section etc.
\section*{Examples}
Chapter~\ref{cha:apple_ref} is based on work conducted in UBC's Maple
Syrup Laboratory by Dr. A.  Apple, Professor B. Boat, and Michael
McNeil Forbes. I was responsible for tapping the trees in forests X
and Z, conducted and supervised all boiling operations, and performed
frequent quality control tests on the product.

A version of chapter~\ref{cha:apple_ref} has been
published~\cite{Apple:2010}. I conducted all the testing and wrote
most of the manuscript. The section on ``Testing Implements'' was
originally drafted by Boat, B.  Check the first pages of this
chapter to see footnotes with similar information.

Note that this preface must come before the table of contents.  Note
also that this section ``Examples'' should not be listed in the table
of contents, so we have used the starred form: \verb|\section*{Example}|.

\tableofcontents                %% Mandatory
\listoftables                   %% Mandatory if thesis has tables
\listoffigures                  %% Mandatory if thesis has figures
\listof{Program}{List of Programs} %% Optional
%% Any other lists should come here, i.e.
%% Abbreviation schemes, definitions, lists of formulae, list of
%% schemes, glossary, list of symbols etc.

\chapter{Acknowledgements}      %% Optional
This is the place to thank professional colleagues and people who have
given you the most help during the course of your graduate work.

\chapter{Dedication} %% Optional
The dedication is usually quite short, and is a personal rather than
an academic recognition.  The \emph{Dedication} does not have to be
titled, but it must appear in the table of contents.  If you want to
skip the chapter title but still enter it into the Table of Contents,
use this command \verb|\chapter[Dedication]{}|.

Note that this section is the last of the preliminary pages (with
lowercase Roman numeral page numbers).  It must be placed
\emph{before} the \verb|\mainmatter| command.  After that, Arabic
numbered pages will begin.

% Any other unusual prefactory material should come here before the
% main body.

% Now regular page numbering begins.
\mainmatter

% Parts are the largest structural units, but are optional.
%\part{Thesis}

% Chapters are the next main unit.
\chapter{This is a Chapter}

% Sections are a sub-unit

\newtheorem{theorem}{Theorem}
\newtheorem{corollary}{Corollary}
\newtheorem{lemma}{Lemma}
\newtheorem{proposition}{Proposition}
\newtheorem{assumption}{Assumption}
 
\newcommand{\C}{\mathbb{C}}
\newcommand{\Tr}{\text{Tr}}
\newcommand{\eps}{\varepsilon}
\newcommand{\R}{\mathbb{R}}
\newcommand{\N}{\mathbb{N}}
\newcommand{\Z}{\mathbb{Z}}
\newcommand{\norm}[1]{\lVert #1 \rVert}
\newcommand{\One}{\mathbbm{1}}
\newcommand{\Var}{\text{Var}}
\newcommand{\F}{\mathcal{F}}
\newcommand{\G}{\mathcal{G}}
\newcommand{\U}{\mathcal{U}}



\section{Introduction}

\section{Nonnteracting Setting}


Consider the lattice $\Z^2$, on which we define a bulk Hamiltonian $H_B$, whose matrix elements follow a short-range assumption:

\[\sup_{y\in\Z^2}\sum_{x\in\Z^2}|H_B(x,y)|(e^{\mu|x-y|}-1) < \infty\]

for some $\mu>0$. We define the bulk conductivity

\[\sigma_B(\lambda) = -i\Tr(P_\lambda[[P_\lambda,\Lambda_1],[P_\lambda,\Lambda_2]])\]

where $P_\lambda$ is the projection onto the eigenstates of $H_B$ with energy lies in $(-\infty,\lambda)$, and where 

\[\Lambda_i(x) = \begin{cases} 1 & x_i < 0\\ 0 & x_i \geq 0\end{cases}\]

are characteristic functions. We construct an edge Hamiltonian on the lattice $\Z^2_a = \{x \in \Z^2 : x_2 > -a\}$. We denote the edge Hamiltonian by $H_a:\ell^2(\Z^2_a) \to \ell^2(\Z^2_a)$, requiring only that that the edge operator $E_a:\ell^2(\Z^2_a)\to\ell^2(\Z^2_a)$ define by 

\[E_a = J_aH_a-H_BJ_a\]

satisfies the edge assumption

\[\sup_{x\in\Z^2}\sum_{y\in\Z^2_a} E_a(x,y)|e^{\mu(|x_2+a|-|x_1-y_1|)} < \infty\]

for some $\mu>0$, where $|x| := |x_1|+|x_2|$. The interpretation





Each site $x \in \mathbb{Z}^2$ get an associated Hilbert space $\mathcal{H}_x$. The dimension of these Hilbert spaces is bounded uniformly in $x$. We consider the Hilbert space $\ell^2(\mathbb{Z}^2, \mathbb{C}^n) = \{ (x_1,x_2,\ldots) \subset \mathbb{C}^n : \sum_{i \in \Z^2} \norm{x_i}^2 < \infty\}$. For example, one might consider a system of spins at the lattice sites, in which case the Hilbert space $\mathcal{H}_x$ at each site would be $\C^2$, and the total Hilbert space $\mathcal{H} = \otimes_x \mathcal{H}_x$ would then be the space of summable wavefunctions $\psi = \otimes_x\psi_x \in \ell^2(\Z^2,\C^2)$.

The Hilbert space $\ell^2(\Z^2)$ is the ``bulk" setting, i.e. the setting in which we consider an infinite two-dimensional medium with no edges, and we consider a ``bulk Hamiltonian" $H_B$ on this Hilbert space. We also define the ``edge" Hilbert space $\ell^2(\Z_a^2)$ and an associated ``edge Hamiltonian" $H_a$, where $\Z^2_a := \{(n,m) \in \Z^2 : n \geq -a\}$. The bulk and edge Hamiltonians are related by the edge operator $E_a : \ell^2(\Z^2_a) \to \ell(\Z^2)$ defined by

\[E_a := J_aH_a - H_BJ_a,\]

where $J_a : \ell^2(\Z_a^2) \to \ell(\Z^2)$ denotes extension by zeroes. We assume that

\begin{assumption}
The edge operator satisfies
\[\sup_{z\in\Z^2}\sum_{y\in\Z^2_a} |E_a(x,y)| e^{\mu(|x_2 + a| + |x_1-y_1|)} < \infty.\]
\end{assumption}



The interpretation is that $E_a = J_aH_a - H_BJ_a$ is the difference between first applying $H_a$ on $\ell^2(\Z^2_a)$, and then making everything below $-a$ into zeroes, versus first making all $x \in \Z^2$ such that $x_2 < -a$ zeroes, and the applying $H_B$. The assumption ensures that the effects from introducing the edge at $-a$ die exponentially as we move upward away from the edge (due to the $|x_2 - (-a)|$ term in the exponent), and also terms do not interact too much as their $x_1$ distance increases (due to the $|x_1-y_1|$ term in the exponent).

We also make the following assumption about both the bulk and edge Hamiltonians:

\begin{assumption}
The Hamiltonians have a spectral gap. There exists an interval $\Delta$ such that $\Delta \cap \sigma(H) = \varnothing$.
\end{assumption}

\textit{Remark}: The spectral gap assumption can be relaxed to a ``mobility gap" assumption,
\[\sup_{f \in B_c(\Delta)}|f(H_B)(x,y)|(1+|x|)^{-\nu}e^{\mu|x-y|} < \infty\]
for some $\nu>0$, where $B_c(\Delta)$ is the set of Borel functions $f$ which are constant on $(-\infty,\inf \Delta)$ and on $(\sup \Delta, \infty)$ such that $|f(x)| \leq 1$ for all $x$. See ? for details.

An example of an edge Hamiltonian satisfying the assumption on $E_a$ is $H_a = J_a^*H_BJ_a$, where $J_a:\ell^2(\Z^2_a)\to\ell^2(\Z^2)$ denotes extension by zeros. The idea is that for a state $\psi \in \ell^2(\Z^2_a)$, we have $\langle \psi, H_a \psi \rangle = \langle (J_a \psi), H_B (J_a\psi) \rangle$, which we interpret as the edge Hamiltonian having the same expectation as the bulk Hamiltonian if we just turned all the states $\psi_x$ with $x_2<-a$ into zeroes.  The edge operator is 

\[E_a = J_aJ_a^*H_BJ_a - H_bJ_a = (J_aJ_a^*-\One)H_BJ_a = \begin{cases} -H_B(x,y) & \text{if } x_2 < -a \\ 0 & \text{if } x_2\geq -a\end{cases}\]

Intuitively, there is no difference between $H_B$ and $H_a$ on $\Z^2_a$. The bound in assumption ? is satisfied by the short range assumption ?.

We define the \textit{bulk conductivity} at Fermi energy $\mu$ as follows. Suppose we subject the system to an external electric potential difference $V$ in the $x_2$ direction. We write this as $-V_0 \Lambda_2$, where $\Lambda_i$ are multiplication operators $\Lambda_i |\psi\rangle = \Lambda(x_i)|\psi\rangle$ which are \textit{switch functions}, satisfying

\[\Lambda(x_i) = \begin{cases} 1 & \text{if } x_i<0 \\ 0 & \text{if } x_i > 1\end{cases}\]

and are smooth and monotonically decreasing on $[0,1]$. This gives $\vec{E}=-\nabla V = V_0\frac{\partial \Lambda_2}{\partial x_2}$, so that $\vec{E}$ is has compact support $\text{supp}(\Lambda_2')$. We introduce a function which grows slowly in time as $t$ grows from $-\infty$ to 0, so as to invoke the adiabatic principle. Here, we choose $e^{\eps t}$, and we will let $\eps\to 0$ at the end. The Hamiltonian therefore experiences a perturbation, 

\[\widetilde{H}_B(t) = H_B - V_0\Lambda_2 e^{\eps t}.\]

We define the Hall current operator $J_H = i[\widetilde{H}_B(t), \Lambda_1] = i[H_B,\Lambda_1]$, which is related to the Hall conductivity by $J_H = \sigma_H V$. 

\begin{proposition}

The Hall conductivity $\sigma_H$ in the bulk system is equal to 

\[\sigma_B = -i\emph{Tr}\left(P_\mu \left[[P_\mu,\Lambda_1],[P_\mu,\Lambda_2]\right]\right),\]

where $P_\mu := P((-\infty,\mu))$ is the projection-valued measure associated with $H_B$ onto states with energy less than the Fermi energy $\mu$.
\end{proposition}
\begin{proof}
We begin with the Heisenberg equation of motion for the density matrix, $\dot{\rho}(t) = - i[\widetilde{H}_B(t),\rho(t)]$, with initial condition $\lim_{t\to-\infty} \|\rho(t) - e^{-itH_B}P_\mu e^{itH_B}\| = 0$, which also implies $\lim_{t\to-\infty} \|e^{itH_B}\rho(t)e^{-itH_B} - P_\mu\| = 0$.

We work in the interaction picture, and define $\rho_I(t) = e^{itH_B}\rho(t)e^{-itH_B}$, and $\Delta H_B(t) = -e^{itH_B}V_0\Lambda_2 e^{\eps t}e^{-itH_B}$. Thus

\[\dot{\rho}_I(t) = -i[\Delta H_B(t), \rho_I(t)]\]

The solution to this differential equation is readily verified to be

\[\rho(t) = i\int_{-\infty}^t [\Delta H_B(t), P_\mu]ds + P_\mu \]

where $\Delta H_B(t)$ depends on $s$. Indeed, taking the derivative of the right hand side gives $i[\Delta H_B(t), P_\mu] = i[\Delta H_B(t), \rho_I(t)] + \mathcal{O}(V_0^2)$, but $P_\mu$ and $\rho_I(t)$ are equal up to zeroth order in $V_0$. The initial condition is also satisfied. 

Using $J_H = i[H_B,\Lambda_1] = \sigma_H V$, we obtain $\sigma_H = \frac{1}{V_0} \lim_{\eps\to 0} \Tr(\rho(0)i[H_B,\Lambda_1])$. Since the expectation of the ground state current is zero, $\Tr(P_\mu J_H) =0$, we have

\[\begin{aligned}
\sigma_H &= \frac{i}{V_0} \lim_{\eps\to 0}\Tr \left( i\int_{\infty}^0 [\Delta H_B(t), P_\mu] [H_B,\Lambda_1] ds\right)\\
&= -\frac{1}{V_0} \lim_{\eps\to 0}\Tr \left( \int_{\infty}^0 [-e^{isH_B}V_0\Lambda_2 e^{\eps s}e^{-isH_B}, P\mu] [H_B,\Lambda_1] ds\right)\\
&= -\lim_{\eps\to 0}\Tr \left( \int_{\infty}^0 e^{isH_B}[\Lambda_2, P_\mu]e^{-isH_B}[H_B,\Lambda_1] e^{\eps s}ds\right)\\
&= -\lim_{\eps\to 0}\Tr \left( \int_{\infty}^0 (e^{-isH_B}[H_B,\Lambda_1]e^{isH_B})\cdot([\Lambda_2, P_\mu] e^{\eps s})ds\right)\\
\end{aligned}\]

Where we used the fact that $P_\mu$ and $H_B$ commute. Using integration by parts on the two terms in brackets, and noting that $\frac{d}{ds}(e^{isH_B}[H_B,\Lambda_1] e^{-isH_B}) = -(ie^{isH_B}\Lambda_1 e^{-isH_B} - \Lambda_1)$, we obtain

\[\begin{aligned}
\sigma_H &= i \lim_{\eps\to 0}\Tr \left( \int_{\infty}^0 (e^{-isH_B}\Lambda_1e^{isH_B} - \Lambda_1)\frac{d}{ds}([\Lambda_2, P_\mu] e^{\eps s})ds\right)\\
&= i \lim_{\eps\to 0}\eps \Tr \left( \int_{\infty}^0 \Lambda_1^s[\Lambda_2, P_\mu] e^{\eps s})ds\right)\\
\end{aligned}\]

where $\Lambda_1^s := e^{-isH_B}\Lambda_1e^{isH_B} - \Lambda_1$. Using the notation $\overline{A} := P_\mu AP_\mu^\perp + P_\mu^\perp A P_\mu$, it is readily verified that the commutator $[\Lambda_2, P_\mu]$ is an \textit{off-diagonal} operator, in the sense that  $[\Lambda_2, P_\mu] = \overline{[\Lambda_2,P_\mu]}$. Furthermore, a simple computation reveals that for any two operators $A$ and $B$, $\Tr(\overline{A}B) = \Tr(A\overline{B})$. It therefore follows that

\[\begin{aligned}
\sigma_H &= i \lim_{\eps\to 0}\eps \Tr \left( \int_{-\infty}^0 \overline{\Lambda_1^s}[\Lambda_2, P_\mu] e^{\eps s})ds\right)
\end{aligned}\]

The integrand can be broken into two terms, 

\[\overline{\Lambda_1^s}[\Lambda_2, P_\mu] e^{\eps s} = e^{-isH_B}\overline{\Lambda_1}e^{isH_B}[\Lambda_2, P_\mu]e^{\eps s} - \overline{\Lambda_1}[\Lambda_2, P_\mu]e^{\eps s}\]

by commutativity of $P_\mu$ and $H_B$. We show that the integral of the first term vanishes. We begin by breaking the first term down further into

\[e^{-isH_B}P_\mu \Lambda_1 P_\mu^\perp e^{isH_B}[\Lambda_2, P_\mu]e^{\eps s} + e^{-isH_B}P_\mu^\perp \Lambda_1 P_\mu e^{isH_B}[\Lambda_2, P_\mu]e^{\eps s}.\]

We treat the first of these two terms; the other is handled in an identical manner. We use the spectral theorem to write $e^{-isH_B}P_\mu = \int_{-\infty}^\mu e^{-is\lambda}dP_\lambda$, and similarly $P_\mu^\perp e^{isH_B} = (\text{Id} - P_\mu) e^{isH_B} = \int_\mu^\infty e^{is\nu} dP_\nu$. 

We remark that, since the Fermi energy $\mu$ is assumed to lie in a spectral gap, there must exist a neighbourhood $(\mu-\delta, \mu+\delta)$ in which there are no states. We exploit this fact to rewrite the limits of integration, $\int_{-\infty}^{\mu-\delta} e^{-is\lambda}dP_\lambda$ and $\int_{\mu+\delta}^\infty e^{is\nu} dP_\nu$. We therefore obtain

\[\begin{aligned} 
\lim_{\eps\to 0} \eps \int_{-\infty}^0 e^{-isH_B}P_\mu \Lambda_1 P_\mu^\perp e^{isH_B}[\Lambda_2, P_\mu]e^{\eps s} ds &= \lim_{\eps\to 0} \eps \Tr\left( \int_{-\infty}^0 \int_{-\infty}^{\mu-\delta} e^{-is\lambda}dP_\lambda \Lambda_1 \int_{\mu+\delta}^\infty e^{is\nu} dP_\nu [\Lambda_2, P_\mu] e^{\eps s} ds\right)\\
&= \lim_{\eps\to 0} \eps \Tr\left( \int_{-\infty}^0 \int_{-\infty}^{\mu-\delta} \int_{\mu+\delta}^\infty e^{s(\eps-i\lambda+i\nu)}dP_\lambda \Lambda_1 dP_\nu [\Lambda_2, P_\mu]ds\right)\\
\end{aligned}\]

Performing the integral over $s$ yields

\[\lim_{\eps\to 0}\eps \int_{-\infty}^0 e^{s(\eps-i\lambda+i\nu)}ds = -\lim_{\eps\to 0} \frac{\eps}{i\eps + \lambda-\nu}\]

This limit is zero, since $\lambda \neq \nu$. Indeed, due to the spectral gap, the integration variables live in $\lambda \in (-\infty, \mu-\delta)$ and $\nu \in (\mu+\delta,\infty)$. The case for the $e^{-isH_B}P_\mu^\perp \Lambda_1 P_\mu e^{isH_B}[\Lambda_2, P_\mu]e^{\eps s}$ term (where the $P_\mu$ and $P_\mu^\perp$ swap places) is treated analogously. Hence the first term in the integrand for $\sigma_H$ vanishes, as claimed. 

Finally, we return to our expression for the Hall conductivity, which now reads

\[\sigma_H = i\lim_{\eps\to 0}\eps \Tr\left(\int_{-\infty}^0 \overline{\Lambda_1}[\Lambda_2, P_\mu]e^{\eps s}ds\right).\]

It is a basic algebraic calculation to show that $\overline{\Lambda_1} = [[\Lambda_1, P_\mu],P_\mu]$. Evaluating the integral over $s$ is now trivial; $\int_{-\infty}^0 e^{\eps s}ds = \eps^{-1}$. Thus

\[\sigma_H = -i\Tr([[\Lambda_1,P_\mu],P_\mu][\Lambda_2,P_\mu]).\]

Shifting the commutator completes the proof:

\[\begin{aligned}
\sigma_H &= -i\Tr(P_\mu[[\Lambda_2,P_\mu],[\Lambda_1,P_\mu]]) \\
&= i\Tr(P_\mu[[\Lambda_1,P_\mu],[\Lambda_2,P_\mu]])\\
&= i\Tr(P_\mu[[P_\mu,\Lambda_1],[P_\mu,\Lambda_2]]).
\end{aligned}\]

\end{proof}

\textit{Remark:} This is reminiscent of the well-known adiabatic curvature formula,

\[\kappa = \Tr(P[\partial_1P,\partial_2 P]) = \Tr(P\left[ [P,K_1], [P,K_2] \right]) = \Tr(P[K_1,K_2]),\]

where $K_i$ are called \textit{generators of parallel transport}. We will see the adiabatic curvature formula again later in the interacting setting. 

For the \textit{edge conductivity}, we need the current operator across the line $x_1=0$, which is given by $-i[H_a,\xi_1]$. We define 

\[\sigma_E = -i\lim_{a\to\infty}\Tr(\rho'(H_a)[H_a,\xi_1]),\]

where $\rho \in C^\infty(\R)$ satisfies

\[\rho(r) = \begin{cases} 1 & \text{if } r<\inf\Delta\\ 0 & \text{if } r>\sup\Delta\end{cases}\]

and decreases smoothly and monotonically in $\Delta$. The definition of $\sigma_E$ is reminiscent of another formula we will see later in the interacting setting, $\Tr(\dot{P}J)$, where $J$ is the current operator. The interpretation of $\sigma_E$ is that if we apply a small potential difference $V$ across $x_2=-a$ to $x_2=\infty$, there will be a net current

\[\begin{aligned}
I &= -i\Tr(\rho(H_a+V)[H_a+V,\xi_1] - \rho(H_a)[H_a,\xi_1])\\
&= -i\Tr((\rho(H_a+V)-\rho(H_a))[H_a,\xi_1])
\end{aligned}\]

Thus we obtain the conductivity

\[\sigma_E = \frac{I}{V} = -i\Tr\left(\frac{(\rho(H_a+V)-\rho(H_a)}{V}[H_a,\xi_1]\right)) \to -i\Tr(\rho'(H_a)[H_a,\xi_1])\]

in the limit as $V\to0$. As we shall see, it turns out that $\sigma_E$ is independent of the choice of $\rho$, and $\sigma_B$ is independent of $\lambda$. 

The main result of this section is

\begin{theorem}
$\sigma_E=\sigma_B$.
\end{theorem}

\subsection{Outline of the Proof}

First, let 

\[\tilde{\sigma}_E(a,t) = -i\Tr(\rho'(H_a)[H_a,\Lambda_1]\Lambda_{2,a}(t))\]

where $\Lambda_{2,a}(t) = e^{itH_a}\Lambda_2 e^{-itH_a}$ is the time evolution of $\Lambda_2$. One can show that, while 

\[\sigma_E = \lim_{T\to\infty}\lim_{a\to\infty} \frac{1}{T}\int_0^T \text{Re}(\tilde{\sigma}_E(a,t))dt,\]

it is unfortunately the case that $\lim_{a\to\infty}\norm{\rho'(H_a)[H_a,\Lambda_1]\Lambda_{2,a}(t)}_1 = \infty$. However, even though the trace norm diverges, it turns out that the trace itself does not, so we will instead subtract a clever choice of zero-trace operator $Z(a,t)$ to define

\[\sigma_E(a,t) = -i\Tr(\rho'(H_a)[H_a,\Lambda_1]\Lambda_{2,a}(t)-Z(a,t))\]

so that the equation $\sigma_E = \lim_{T\to\infty}\lim_{a\to\infty} \frac{1}{T}\int_0^T \text{Re}(\sigma_E(a,t))dt$ still holds, but we also have $\lim_{a\to\infty}\norm{\rho'(H_a)[H_a,\Lambda_1]\Lambda_{2,a}(t) - Z(a,t)}_1 < \infty$. The correct choice of $Z$ will become apparent after writing $\rho(H_a)$ and $\rho'(H_a)$ in terms of their Hellfer-Sjostrand representations,

\[\rho(H_a) = \frac{1}{2\pi} \int_\C \frac{\partial \tilde{\rho}(z)}{\partial \bar{z}} R(z)\]

\[\rho'(H_a) = -\frac{1}{2\pi}\int_\C \frac{\partial \tilde{\rho}(z)}{\partial \bar{z}} R(z)^2\]

where $R(z) = (H_a - z)^{-1}$ is the resolvent. Using $[R(z),\Lambda_i] = R(z)[H_a,\Lambda_i]R(z)$, we obtain the representations of the following useful operators:

\[[\rho(H_a),\Lambda_1] = \frac{1}{2\pi} \int_\C \frac{\partial \tilde{\rho}(z)}{\partial \bar{z}} R(z)[H_a,\Lambda_1]R(z)\]

\[\rho'(H_a)[H_a,\Lambda_1] = -\frac{1}{2\pi} \int_\C \frac{\partial \tilde{\rho}(z)}{\partial \bar{z}} R(z)^2[H_a,\Lambda_1]\]

From here, we define the zero-trace operator

\[Z(a,t) = [\rho(H_a),\Lambda_1]\Lambda_2 - \frac{1}{2\pi} \int_\C \frac{\partial \tilde{\rho}(z)}{\partial \bar{z}} R(z)(R(z)[H_a,\Lambda_1]\Lambda_{2,a}(t) - [H_a,\Lambda_1]\Lambda_{2,a}(t)R(z))\]

from which we obtain

\[\begin{aligned}
\sigma_E(a,t) &= \tilde{\sigma}_E(a,t) - Z(a,t)\\
&= \Tr\left(-[\rho(H_a),\Lambda_1]\Lambda_2 - \frac{1}{2\pi}\int_\C \frac{\partial \tilde{\rho}(z)}{\partial \bar{z}}R(z)[H_a,\Lambda_1]\Lambda_{2,a}(t)R(z)\right)\\
&= \Tr\left([\rho(H_a),\Lambda_1](\Lambda_{2,a}(t)-\Lambda_2) - \frac{1}{2\pi}\int_\C\frac{\partial \tilde{\rho}(z)}{\partial \bar{z}}R(z)[H_a,\Lambda_1]R(z)[H_a,\Lambda_{2,a}(t)]R(z)\right)
\end{aligned}\]

All of the statements so far can be verified by calculations. The difficult part of the theorem (aside from proving that the relevant operators are trace-class) is proving that

\[\norm{J_a\Sigma_a'J_a^* - \Sigma_B'}_1, \norm{J_a\Sigma_a''J_a^* - \Sigma_B''}_1 \to 0\]

as $a\to\infty$, where $\Sigma_B'$ and $\Sigma_B''$ are the same as with the subscript $a$, except using the bulk Hamiltonian $H_B$ in their definition rather than $H_a$. It follows that 

\[\sigma_E(a,t) = \Tr(J_a\Sigma_a'J_a^*+J_a\Sigma_a''J_a^*) = \Tr(\Sigma_a'+\Sigma_a'') \to \Tr(\Sigma_B'+\Sigma_B'')\]

From there, a calculation shows that $\lim_{T\to\infty}\frac{1}{T}\int_0^T\Tr(\Sigma_B'+\Sigma_B'')dt = \sigma_B$, concluding the proof.

\section{Interacting Setting}

Let $L\in\N$, and let $\Lambda_L = \Z_L \times [0,L]$ be the discrete cylinder, equipped with a metric $d$. To each site $x \in \Lambda_L$, we associate a Hilbert space $\mathcal{H}_x$ whose dimension is bounded uniformly in $L$, i.e. there exists some $N>0$ such that for all $L \in \N$, we have $\text{dim}(\mathcal{H}_x) \leq N$ for all $x \in \Lambda_L$. For a subset $X \subseteq \Lambda_L$, we define the Hilbert space $\mathcal{H}_X = \otimes_{x \in X} \mathcal{H}_x$, and we set $\mathcal{H}_L := \mathcal{H}_{\Lambda_L} = \otimes_{x \in \Lambda_L}\mathcal{H}_x$. For simplicity, throughout we take $L = 2^n$ for some large $n$.

The algebra $\U_L \subset \mathcal{B}(\mathcal{H}_L)$ of observables on $\Lambda_L$ is the set of bounded operators on $\mathcal{H}_L$ which are self-adjoint. For an operator $A_X \in \U_X$, we identify its extension to $\U_L$ by taking its tensor product with copies of the identity, $A_X \otimes_{x \in X^\mathsf{c}} \mathbb{I}_x$. Conversely, we say that an operator $A_X \in \Lambda_L$ has support $X$ if $A_X = A_X|_{X} \otimes_{x \in X^\mathsf{c}} \mathbb{I}_x$, and write $A_X \in \U_X$.

A \textit{local interaction} is a map $\Phi : \mathcal{P}(\Lambda_L) \to \U_L$ such that $\Phi(X) = 0$ whenever $\text{diam}(X) > R$, $\Phi(X)$ is supported in $X$, and $\norm{\Phi(X)} \leq C$ for all $X \subset \Lambda_L$, for all $L$.

We consider a region as depicted in Figure~\ref{fig:setup}. In the left white region $[0,L/2] \times [0,L]$, $H_0$ is a trivial Hamiltonian which we take to be the void (for example $H_0=0$), and in the right blue region $[L/2,L]\times [0,L]$, $H_1$ is a nontrivial \textit{local Hamiltonian}, in the sense that $H_1 = \sum_{X \subseteq \Lambda} \Phi(X)$, where $\Phi$ is a local interaction. We define the Hamiltonian of the full system to be 

\[H(\mu) = H_1 + \mu Q_h,\]

where $Q_h = \sum_{x \in \Gamma_h} a_x^*a_x$ is the number operator for the region $\Gamma_h = [L/4,3L/4] \times [0,L]$. 

\begin{figure}[h!]
\centering
\begin{tikzpicture}
\fill[color=blue!30] (7.5,0) rectangle (15,10);
\draw[gray, thick] (7.5,0) -- (7.5,10);
\fill[pattern=north east lines, pattern color=red!30] (3.75,0) rectangle (11.25,10);
\fill[pattern=north west lines, pattern color=yellow!30] (0,5) rectangle (15,10);
\draw[blue, thick] (0,0) rectangle (15,10);
\draw[dashed, color=yellow] (0,5) -- (15,5);
\draw[dashed, color=red] (3.75,0) -- (3.75,10);
\draw[dashed, color=red] (11.25,0) -- (11.25,10);
\node[] at (1,1) {$H_0$};
\node[] at (14,1) {$H_1$};
\node[color=red] at (10.25,4) {$\Gamma_h$};
\node[color=yellow] at (10.25,9) {$\Gamma_u$};
\end{tikzpicture}
\caption{The setup for $\Lambda_L$. The left and right edges are identified to form a cylinder.}
\label{fig:setup}
\end{figure}

We also consider the plane $\Z^2$. In this setting, there are no edge states, and so the associated ``bulk" Hamiltonian $H_B(\mu)$ is assumed to have a \textit{gapped} spectrum, in the sense that

\begin{assumption}
\[\sigma(H_B) = \mathcal{S}_{-} \cup \mathcal{S}_{+},\]
where $\inf\mathcal{S}_{+} - \sup \mathcal{S}_{-} \geq \gamma$ uniformly in $L$ and $\mu$ for some $\gamma > 0$. 
\label{ass:gap}
\end{assumption}

In the case of the cylinder, this effect does not necessarily occur due to the presence of the edge. 
We also assume that the Hamiltonian is \emph{charge-conserving}.

\begin{assumption}
$[H(\mu), Q] = 0$, where $Q$ is the total charge in $\Lambda_L$.
\label{ass:charge}
\end{assumption}

Let $P_B$ be the ground state projection of $H_B$ (the system without an edge), and let $P$ be the ground state projection of $H$ (the system with an edge). We assume that states far from the edge are essentially bulk states, up to tails that vanish quickly in $L$.

\begin{assumption}
Define the bulk region $\Gamma_B = [L/2+k, L] \times [0,L]$ for some $k > 2R$. For any operator $A \in \U_{\Gamma_B}$,
\[\emph{Tr}(PA) = \emph{Tr}(P_BA) + \mathcal{O}(L^{-\infty}).\]
\label{ass:bulk}
\end{assumption}

(need to add justification)

For ease of notation, we omit the subscript $L$ wherever there is no risk of confusion.

\section{Equality of Bulk and Edge Currents}

\subsection{Cylinder Geometry}

Let $P(\mu)$ be the (possibly degenerate) ground state projection of $H(\mu)$. Let $Q_u = \sum_{x \in \Gamma_u} a_x^* a_x$ be the charge in the upper half of the cylinder $\Gamma_u = [0,L] \times [L/2,L]$, and define current operator 

\[J_u = i[H(\mu),Q_u].\]

For simplicity, we drop the subscript $u$ and simply write $J=J_u$. Charge conservation~\ref{ass:charge} implies that this current operator is supported on the strip $[L/2, L] \times [L/2-R, L/2+R]$, i.e. along a strip of width $2R$ centred at the line $y=L/2$. Indeed, if we inspect a local interaction $\Phi(X)$ of range $R$ with support $S \subset (\Gamma_u)_R$, where $(\Gamma)_\alpha$ is the $\alpha$-shrinking of $\Gamma$, then clearly $[\Phi(X), Q_u] = [\Phi(X), Q] = 0$, by assumption~\ref{ass:charge}. Similarly, if $\Phi(X)$ has support $S \subset ((\Gamma_u)^\mathsf{c})_R$, then $[\Phi(X), Q_u] = [\Phi(X), Q] = 0$. It follows that for an interaction $\Phi(X)$ with range $R$ and arbitrary support, $[\Phi(X), Q_u]$ must be supported on a set which is contained in (or equal to) the strip $[L/2, L] \times [L/2-R, L/2+R]$, and so $[H,Q_u]$ must be as well, since $H$ is a sum of such local interactions.

% In the case of a unique ground state $\Omega$, the total change in charge in $\Gamma_h$ after threading one quantum of flux is given by the fundamental theorem of calculus,

%\[ \int_{t:\Phi = 0}^{t:\Phi=2\pi} i[H, Q_u] dt' = \langle \Omega, (U^* Q_h U - Q_h) \Omega \rangle\]

\begin{lemma}
The ground state expectation of the current $J$ is zero.
\label{lemma:J=0}
\end{lemma}
\begin{proof}
By cyclicity of the trace and commutativity of $P(\mu)$ with $H(\mu)$, 
\[\begin{aligned}
\Tr(P(\mu) J) &= \Tr(P(\mu)i[H(\mu),Q_u]) \\
&= \Tr(iP(\mu)H(\mu)Q_u) - \Tr(iP(\mu)Q_u H(\mu))\\
&= \Tr(iH(\mu)P(\mu)Q_u) - \Tr(iH(\mu)P(\mu)Q_u)\\
&=0.
\end{aligned}\]
\end{proof}

Define the operators

\[K(\mu) = \mathcal{I}_{\mu}(\dot{H}(\mu)),\]

where 

\[\mathcal{I}_{\mu} (A) = \int_\R W(t) e^{itH(\mu)} A e^{-itH(\mu)}dt.\]

More explicitly, in this setting we see that $K(\mu) = \mathcal{I}_\mu(Q_h)$. As a shorthand, we use the notation $\dot{A}(\mu_0)= \frac{d}{d\mu}A|_{\mu_0}$. 

We present two important properties of the map $\mathcal{I}_\mu$ in the following lemmas, and leave their proofs to the appendix (need to add). We will also need a definition: an \emph{off-diagonal} operator is an operator $A$ such that $A = \overline{A} := PAP^\perp + P^\perp AP$, where $P^\perp = \mathbb{I} - P$ is the projection onto the excited states above the gap.

\begin{lemma}
For any off-diagonal operator $A$, $\mathcal{I}_\mu(\cdot)$ and $[H(\mu), \cdot]$ act as inverses of each other, up to a factor of $i$:

\[\mathcal{I}_\mu\left([H(\mu), A]\right) = [H(\mu), \mathcal{I}_\mu(A)] = iA\]
\label{lemma:inverseofH}
\end{lemma}

It is easy to verify that any operator $A$ behaves as an off-diagonal operator when taking a commutator with $P$, in the sense that

\[[\overline{A}, P] = [PA(1-P) + (1-P)AP, P] = [A,P].\]

It follows that for any (not necessarily off-diagonal) operator $A$, 

\[[\mathcal{I}([H, A]),P] = i[A,P]\]

\begin{lemma}
$\mathcal{I}_\mu$ is local in the sense that for any $A \in \mathcal{U}_X$, 

\[\norm{\mathcal{I}(A)_{(X^r)^\mathsf{c}}} \leq \norm{A} |X| \mathcal{O}(r^{-\infty}).\]
\label{lemma:local}
\end{lemma}

\begin{proposition}
The operator $K(\mu)$ is the \emph{generator of parallel transport}, satisfying
\[\dot{P}(\mu) = i[K(\mu),P(\mu)].\]
\label{prop:generatorparalleltransport}
\end{proposition}
\begin{proof}
By the product rule and the fact that $H(\mu)$ and $P(\mu)$ commute, 

\[ [\dot{H}(\mu), P] = -[H, \dot{P}(\mu)]. \]

Now we show that $\dot{P}(\mu)$ is off-diagonal. Taking the derivative on both sides of $P^2=P$, we see that $\dot{P}P + P\dot{P} = \dot{P}$. Acting on the left and right with $P$ on both sides of this equation gives 

\[P\dot{P}P + P\dot{P}P = P\dot{P}P,\]

which implies that $P\dot{P}P = 0$. Now, 

\[\begin{aligned}
\overline{\partial_\mu P} &= P(\partial_\mu P)(1-P) + (1-P)(\partial_\mu P)P\\
&= P(\partial_\mu P) - P(\partial_\mu P)P + (\partial_\mu P)P - P(\partial_\mu P)P\\
&= P(\partial_\mu P) + (\partial_\mu P)P\\
&= \partial_\mu (P^2)\\
&= \partial_\mu P,
\end{aligned}\]

as claimed. It therefore follows from Lemma~\ref{lemma:inverseofH} that

\[\begin{aligned}
\dot{P}(\mu) &= -i \mathcal{I}_\mu([H(\mu), \dot{P}(\mu)])\\
&= i\mathcal{I}_\mu([\dot{H}(\mu), P(\mu)])\\
&= i[\mathcal{I}_\mu(\dot{H}(\mu)), P(\mu)]\\
&= i[K(\mu), P(\mu)].
\end{aligned}\]
\end{proof}

Increasing the ``electric potential" by a small amount $d\mu Q_h$ and expanding to linear order, the change in ground state current is given by

\[\Tr(P(\mu+d\mu)J) - \Tr(P(\mu)J) = \kappa d\mu + \mathcal{O}(d\mu^2).\]

Dividing by $d\mu$ and taking a limit, we see that the linear response coefficient is given by

\[\kappa = \Tr\left(\dot{P}(\mu) J\right).\]

The \emph{Hall conductivity} of the system on a subset $V \subseteq \Lambda$ is defined to be $\kappa_V := \Tr\left(\dot{P}(\mu) J_V\right)$, where $J_V$ is the restriction of the operator $J$ to $V$. 

\begin{proposition}
The Hall conductivity is independent of the driving strength $\mu$.
\end{proposition}
\begin{proof}
We see by cyclicity of the trace and the formula $\dot{H}(\mu)=Q_h$ that for any $\mu_1$ and $\mu_2$,

\[\begin{aligned}
\kappa(\mu_1) - \kappa(\mu_2) &= \Tr\left( \dot{P}(\mu_1) i[H(\mu_1), Q_u] - \dot{P}(\mu_2)i[H(\mu_1),Q_u]\right)\\
&= i\Tr\left(\left([\dot{P}(\mu_1), H(\mu_1)] - [\dot{P}(\mu_2), H(\mu_2)]\right)Q_u\right)\\
&= i\Tr\left( \left([\dot{H}(\mu_1), P(\mu_1)] - [\dot{H}(\mu_2), P(\mu_2)]\right)Q_u \right)\\
&= i\Tr\left( [Q_h, P(\mu_1) - P(\mu_2)]Q_u \right)\\
&= i\Tr \left( Q_h(P(\mu_1) - P(\mu_2))Q_u - (P(\mu_1)-P(\mu_2))Q_uQ_h \right)\\
&= 0,
\end{aligned}\]

since $Q_h$ and $Q_u$ commute, indicating that the Hall conductivity is independent of $\mu$ as one would expect physically.
\end{proof}

The following is the main result:

\begin{theorem}
The ground state current in the strip $[L/2+k, 3L/4-k] \times [0,L]$ between the edge and the bulk current vanishes, in the sense that $\kappa_V = \mathcal{O}(r^{-\infty}) + \mathcal{O}(L^{-\infty})$ for any $V \subseteq [L/2+R, 3L/4-R] \times [0,L]$ ``in between" the bulk and edge strips, where 

\[r = \emph{dist}(V, [L/2-R, 3L/4+R] \times [0, L] \cup [3L/4-R, 3L/4+R] \times [0, L])\]

is the distance from $V$ to one of the edge or bulk strips.

\end{theorem}

\begin{proof}
By Proposition~\ref{prop:generatorparalleltransport}, the Hall conductivity can also be written by the formula $\kappa_V^B = \Tr\left(i[K(\mu),P_B(\mu)]J_V^B\right) = \Tr\left(i[\mathcal{I}_\mu(Q_h), P_B(\mu)]J_V^B\right)$, where $J_V^B = \left(i[H_B, Q_u]\right)_V$ is the current arising from the bulk Hamiltonian. From commutativity of $P_B$ and $H_B$ along with cyclicity of the trace, we compute

\[\begin{aligned}
\kappa_V^B &= \Tr\left(i[\mathcal{I}_\mu(Q_h), P_B(\mu)]J_V^B\right)\\
&= \Tr\left(i\int_\R W(t) e^{itH_B(\mu)} [Q_h,P_B(\mu)] e^{-itH_B(\mu)}dt J_V^B \right)\\
&= \int_\R W(t) \Tr\left(i[Q_h,P_B(\mu)] e^{-itH_B(\mu)}J_V^Be^{itH_B(\mu)}\right)dt\\
&= -\int_\R W(t) \Tr\left(i[Q_h,P_B(\mu)] e^{itH_B(\mu)}J_V^Be^{-itH_B(\mu)}\right)dt\\
&=- \Tr\left(i[Q_h,P_B(\mu)]\mathcal{I}_\mu(J_V^B)\right),
\end{aligned}\]

since $W(t)$ is odd. Again by cyclicity of trace combined with the fact that $\mathcal{I}_\mu(\cdot)$ is an inverse of $[H_B(\mu), \cdot]$ for commutators with $P_B(\mu)$ (by the remark after lemma~\ref{lemma:inverseofH}), we obtain

\[\begin{aligned}
\kappa_V^B &=-\Tr([\mathcal{I}_\mu([H_B(\mu),Q_h]), P_B(\mu)]\mathcal{I}_\mu(J_V^B))\\
&=-\Tr(\mathcal{I}_\mu([H_B(\mu),Q_h]) P_B(\mu)\mathcal{I}_\mu(J_V^B) - P_B(\mu) \mathcal{I}_\mu([H_B(\mu),Q_h])\mathcal{I}_\mu(J_V^B)) )\\
&=-\Tr(P_B(\mu)\mathcal{I}_\mu(J_V^B) \mathcal{I}_\mu([H_B(\mu),Q_h]) - P_B(\mu) \mathcal{I}_\mu([H_B(\mu),Q_h])\mathcal{I}_\mu(J_V^B)) )\\
&= \Tr\left( P_B(\mu)[\mathcal{I_\mu}([H_B(\mu),Q_h]), \mathcal{I}_\mu(J_V^B)]\right).
\end{aligned}\]

Now, $[H_B(\mu), Q_h]$ is a local operator, supported on the ``bulk line" $[3L/4-R, 3L/4+R] \times [0, L]$, while $J_V^B$ is a local operator supported on $V \subseteq [L/2+k, 3L/4-k] \times [0,L]$. Since $\mathcal{I}_\mu$ preserves locality up to tails, in the sense that $\norm{\mathcal{I}_\mu(A)_{(S^r)^\mathsf{c}}} \leq \norm{A} |S| \mathcal{O}(r^{-\infty})$ (Lemma~\ref{lemma:local}), it follows that the commutator $[\mathcal{I_\mu}([H_B(\mu),Q_h]), \mathcal{I}_\mu(J_V^B)] = C\mathcal{O}(r^{-\infty})$ whenever $V \cap ([3L/4-R, 3L/4+R] \times [L/2, L]) = \varnothing$.

The previous fact applies to the bulk setting with $H_B$ and $P_B$. To extend this to the setting with an edge, it is enough to use Assumption~\ref{ass:bulk} to conclude the same result, except with equality up to $\mathcal{O}(L^{-\infty})$, i.e.

\[\kappa_V = \Tr\left(\dot{P}J_V\right) = \Tr\left(\dot{P}(J_V^B + \mathcal{O}(L^{-\infty}))\right) = \kappa_V^B + \mathcal{O}(L^{-\infty}) = \mathcal{O}(r^{-\infty}) + \mathcal{O}(L^{-\infty}).\]

\end{proof}


The intuitive picture from the previous result is that, in the bulk region, the Hall conductivity is essentially only nonzero along  the bulk line $[3L/4-R, 3L/4+R] \times [L/2, L]$. Since the ground state expectation of the current is zero (by lemma~\ref{lemma:J=0}), it must be that there is an equal current flowing along the edge, but in the opposite direction, see figure (need to add).

\subsection{Torus Geometry}

Our goal is to show the same result on the discrete torus $\mathbb{T}_L := \Z_L \times \Z_L$. We define the same regions $\Gamma_u$ and $\Gamma_h$, and the same current operator $J_u = i[H(\mu), Q_u]$. This time, however, Lemma~\ref{lemma:J=0} does not apply. Intuitively, it does not apply because electrons can now flow through both the bottom and the top of the region $\Gamma_u$, rather than just the bottom. Mathematically, the lemma fails because our definition of the current is slightly changed.

We use charge conservation and the fact that $H$ is finite range to split the current $J_u$ into two components, $J_u = i[H_-, Q_u] + i[H_+, Q_u] = J_- - J_+$, supported on strips of width $2R$ at $y=L/2$ and $y=L$, respectively. We then define the current operator to be $J=J_-$, which is the current on the lower strip. This is the mathematical reason that the proof in Lemma~\ref{lemma:J=0} fails on the torus; we have replaced $H$ by $H_-$, which may no longer commute with $P$. We instead proceed by a different approach. We will need a few auxiliary results first.

\begin{lemma}
$K_\pm$ is supported on $\partial_\pm$ up to tails.
\label{SupportOfK}
\end{lemma}
\begin{proof}

\end{proof}

\begin{proposition}
The operator $Q_h-K$ leaves the ground state space invariant, i.e. $[Q_h-K, P] = 0$.
\end{proposition}
\begin{proof}

\end{proof}


\begin{lemma}
Show that $\emph{Tr}(A,[Q_h,P])=0$ for all $A \in \mathcal{U}_{\text{edge}}$. This shows that $Q_h$ commutes with $P$ ``along the edge".
\label{lemma:[Q,P]=0}
\end{lemma}
\begin{proof}

Let $A \in \mathcal{U}_{\text{edge}}$. Since $H$ is charge conserving, we may choose a simultaneous eigenbasis of $H$ and the total charge $Q$, in which case $P$ and $Q$ commute. It follows that

\[\begin{aligned}
\Tr(A[Q_h, P]) = \Tr([A, Q_h] P) = \Tr([A, Q]P) = \Tr(A[Q,P]) = 0.
\end{aligned}\]



%\[\begin{aligned}
%\Tr(A [Q_h, P]) &= \Tr([A, Q_h]P)\\
%&= \Tr([P,A]Q_h)\\
%&= \sum_{x \in \Gamma_h} \Tr([P, A] c_x^*c_x)\\
%&= \sum_{x \in \Gamma_h} \Tr\left(\bigotimes_{y \in \text{supp}([P, A])} \left[P, A\right]_y c_x^*c_x\right)\\
%&= \sum_{x \in \Gamma_h} \Tr\left(\bigotimes_{y \in \text{supp}([P, A])} \left[P_y, A_y\right] c_x^*c_x\right)\\
%&= \sum_{x \in \Gamma_h} \prod_{y \in \text{supp}([P, A])} \Tr([P_y, A_y] c_x^*c_x)
%\end{aligned}\]

%The support of $[P, A]$ is the edge, so

%\[\begin{aligned}
%\Tr(A [Q_h, P]) &= \sum_{x \in \Gamma_h} \prod_{y \in \text{edge}} \Tr([P_y, A_y] c_x^*c_x)\\
%&= \sum_{x \in \Gamma_h} \prod_{y \in \text{edge}} \Tr((\mathbb{I}\otimes\ldots\otimes\mathbb{I}\otimes [P_y, A_y] \otimes\mathbb{I}\ldots\otimes\mathbb{I}) (\mathbb{I}\otimes \ldots \otimes\mathbb{I} \otimes c^*_xc_x \otimes \mathbb{I} \ldots\otimes\mathbb{I}))\\
%&= \sum_{x \in \Gamma_h} \prod_{y \in \text{edge}} \Tr(\mathbb{I}\otimes\ldots\otimes\mathbb{I}\otimes c_x^*c_x \otimes \mathbb{I} \otimes \ldots \otimes\mathbb{I} \otimes [P_y, A_y] \otimes\mathbb{I}\ldots)\\
%&= \sum_{x \in \Gamma_h} \prod_{y \in \text{edge}, \; y\neq x} \Tr(c_x^*c_x)\Tr([P_y, A_y]) + \sum_{x \in \Gamma_h \cap \text{edge}} \Tr(c_x^*c_x [P_x, A_x]).\\
%\end{aligned}\]

%Since the trace of any commutator is zero, the terms in the first sum vanish. Since $\Gamma_h \cap \text{edge} =\text{edge}$, we are left with

%\[\Tr(A[Q_h, P]) = \sum_{x \in \text{edge}} \Tr(c_x^*c_x[P_x, A_x]) = \sum_{x \in \text{edge}} \Tr(P_x[A_x, c_x^*c_x]).\]

%For any particular $x\in \text{edge}$, write $P_x = \sum_n |\psi_n \rangle \langle \psi_n |$, where the sum is over ground states, and 

%\[\langle \psi | [P_y, A_y]|\psi\rangle = \sum_n \langle \psi | \psi_n \rangle \langle\psi_n | A |\psi\rangle - \sum_n \langle \psi | A |\psi_n\rangle\langle\psi_n | \psi\rangle \]

%Want to show:

%\[\Tr([P_y, A_y]) = \ldots = 0\]

\end{proof}

Finally, we will prove that in the bulk system with Hamiltonian $H_B(\mu)$, the ground state expectation of the current vanishes faster than any power as $L \to \infty$.

\begin{lemma}
The ground state expectation of the current $J_B := i[(H_B)_-, Q_h]$ (of the system without an edge) is $\emph{Tr}(P_BJ_B)=\mathcal{O}(L^{-\infty})$.
\label{J=0Bulk}
\end{lemma}
\begin{proof}
First, $K = \mathcal{I}(i[H_B, Q])$ splits into $K = K_- - K_+$, with the support of $K_\pm$ contained in $\partial_\pm$ up to tails:

\[[K_\pm, A_X] = \mathcal{O}(p^{-\infty}),\]

for every $A_X \in \mathcal{U}_X$ such that $\norm{A_X}=1$, and where $p = \text{dist}(X, \partial_\pm)$ (need to add). Using the fact that $K_\pm$ is supported in $\partial_\pm$ up to tails (Lemma~\ref{SupportOfK}), we see that 

\[i[H_B, K_-] = i[(H_B)_-, K_-] + \mathcal{O}(L^{-\infty}),\]

and similarly $i[(H_B)_-, K_+] = \mathcal{O}(L^{-\infty})$. Putting these facts together, it follows that the current can be rewritten as 

\[\begin{aligned} 
J _B&= i[H_B, Q_h + K_- - K_- + K_+] + \mathcal{O}(L^{-\infty})\\
&= i[H_B, K_-] + i[(H_B)_-, Q_h - K_- + K_+)] + \mathcal{O}(L^{-\infty}).
\end{aligned}\]

From here, we use the fact that $H_B$ and $Q_h-K_-+K_+$ both commute with $P_B$ to write

\[P_BJ_BP_B = i[H_B, P_BK_-P_B] + i[P_B(H_B)_-P_B, Q_h - K_- + K_+)] + P_B\mathcal{O}(L^{-\infty})P_B.\]

Since the trace of any commutator is zero, 

\[\Tr(P_BJ_B) = \Tr(P_BJ_BP_B) = \mathcal{O}(L^{-\infty}).\]

\end{proof}

Using this, we can show a simple proof of the analogue of Lemma~\ref{lemma:J=0} on the torus, in the case of non-interacting systems.

\begin{proposition}
Let $H = \sum_{x \in \mathbb{T}} h_x$ be a non-interacting Hamiltonian, i.e. a sum of single site Hamiltonians $h_x$. The ground state expectation of the current $J=i[H_-, Q_h]$ (of the system with an edge) is $\emph{Tr}(PJ)=\mathcal{O}(L^{-\infty})$.
\label{prop:J=0Torus}
\end{proposition}

\begin{proof}

Since $H$ is a sum of single site Hamiltonians, we can split $H_-$ into the restrictions $H_- = (H_-)_\text{edge} + (H_-)_\text{bulk}$, with no fear of any terms which are in both the edge region and the bulk region. By Assumption~\ref{ass:bulk},

\[\begin{aligned}
\Tr(PJ) &= \Tr(Pi[H_-, Q_h])\\
&= i\Tr([H_-, Q_h]P)\\
&= i\Tr((H_-)_\text{edge} [Q_h, P]) + i \Tr((H_-)_\text{bulk} [Q_h, P])\\
&= i\Tr((H_-)_\text{edge} [Q_h, P]) + i \Tr((H_-)_\text{bulk} [Q_h, (P)_\text{bulk}])\\
&= i\Tr((H_-)_\text{edge} [Q_h, P]) + i \Tr((H_B)_-[Q_h, P_B]) + \mathcal{O}(L^{-\infty})\\
&= i\Tr((H_-)_\text{edge} [Q_h, P]) + \Tr(i[(H_B)_-, Q_h]P_B) + \mathcal{O}(L^{-\infty}).
\end{aligned}\]

By Lemma~\ref{lemma:[Q,P]=0}, the first term is zero. By Lemma~\ref{J=0Bulk}, the second term is $\mathcal{O}(L^{-\infty})$. 
\end{proof}

\newpage
\appendix

\section{Properties of $\mathcal{I}_\mu$}

\begin{proof}
(Of Lemma~\ref{lemma:inverseofH}). Let $\widehat{W}(\xi) = \frac{1}{\sqrt{2\pi}}\int_\R W(t) e^{-2\pi i t \xi} dt$ be the Fourier transform of $W$. One can show that for $|\xi|\geq \gamma$, $\widehat{W}(\xi) = \frac{1}{\sqrt{2\pi}i \xi}$ (need to add). Let $A$ be an observable. First, we show that $\mathcal{I}([H,PAP^\perp]) = iPAP^\perp$. 

Decomposing 

\[\begin{aligned}
e^{itH}P &= \sum_{j=0}^\infty \frac{(itH)^j}{j!} P \\
&= \sum_{j =0}^\infty \frac{(it)^j}{j!} \left(\sum_n E_n^j P_n\right)P \\
&= \sum_{j=0}^\infty \frac{(it)^j}{j!} \sum_{n : E_n=0} E_n^j P_n \\
&= \sum_{n : E_n=0} e^{itE_n}P_n,
\end{aligned}\]

and similarly 

\[P^\perp e^{-itH}= \sum_{m:E_m \geq \gamma}P_m e^{-itE_m},\]

we see that

\[\begin{aligned}
\mathcal{I}([H,PAP^\perp]) &= \mathcal{I}(P[H,A]P^\perp) \\
&= \int_\R W(t)e^{itH}P[H,A]P^\perp e^{-itH}dt \\
&= \int_\R W(t) \sum_{n : E_n=0} e^{itE_n}P_n [H,A] \sum_{m:E_m \geq \gamma} P_m e^{-itE_m} dt\\
&= \sum_{n : E_n=0} \sum_{m:E_m \geq \gamma} \int_\R W(t) e^{itE_n} P_n A (E_n-E_m) P_m e^{-itE_m}dt\\
&= \sum_{n : E_n=0} \sum_{m:E_m \geq \gamma} P_n A P_m (E_n-E_m) \int_\R W(t) e^{-it(E_m - E_n)} dt\\
&= \sum_{n : E_n=0} \sum_{m:E_m \geq \gamma} P_n A P_m (E_n-E_m) \sqrt{2\pi} \widehat{W}(E_m-E_n) \\
&= i\sum_{n : E_n=0} \sum_{m:E_m \geq \gamma} P_n A P_m\\
&= iPAP^\perp.
\end{aligned}\]

(need to check the $2\pi$ factor)

By the same argument, $\mathcal{I}([H,P^\perp AP]) = iP^\perp AP$ as well, and so $\mathcal{I}([H,\overline{A}]) = i\overline{A}$.
\end{proof}

\begin{proof}

(Of Lemma~\ref{lemma:local}). We break the integral into two parts,

\[\norm{\mathcal{I}(A)} \leq \bigg\lVert\int_{-T}^T W(t) e^{itH}Ae^{-itH}dt \bigg\rVert + \bigg\lVert \int_{\R\setminus [-T,T]} W(t) e^{itH}Ae^{-itH}dt\bigg\rVert.\]

The first term can be estimated using the Lieb-Robinson bound found in Appendix~\ref{sec:L-R}.

\end{proof}

\section{Lieb-Robinson Bound}
\label{sec:L-R}

Let $N$ be a uniform upper bound for the dimensions of the Hilbert spaces at each site, i.e. $\text{dim}(\mathcal{H}_x) \leq N$ for all sites $x$.

The following is a version of the Lieb-Robinson. For any operators $A \in \mathcal{U}_X$ and $B \in \mathcal{U}_Y$ having disjoint supports $X\cap Y = \varnothing$, 

\[\norm{ [e^{itH}Ae^{-itH}, B] } \leq  C \norm{A}\norm{B} |X| |Y| N^{2|X|} e^{2t\norm{\Phi}_\lambda - \lambda d(X, Y)}.\]

\section{Gr\"{o}nwall's Inequality and Uniqueness}

\begin{theorem}
(Gr\"{o}nwall's Inequality). Let $\alpha : I \to (0,\infty)$ be positive and continuous on $I^o$ for some interval of the form $[a,b)$, $[a,b]$, or $[a,\infty)$. Suppose $u : \mathbb{R} \to \mathcal{U}$ is a Banach-valued, differentiable function. If $\norm{u'(t)} \leq \alpha(t)\norm{u(t)}$ for all $t\in I$, then 
\[\norm{u(t)} \leq \norm{u(a)} e^{\int_a^t \alpha(s)ds} \quad \forall t\in I\]
\end{theorem}
\label{thm:gronwallinequality}
\begin{proof}
Let $f(t) = e^{\int_a^t \alpha(s)ds}$, which is nonzero, has initial value $f(a)=1$, and has derivative $f'(t) = \alpha(t) f(t)$. Then by the quotient rule,
\[\left(\frac{\norm{u(t)}}{f(t)}\right)' = \frac{\norm{u'(t)}f(t) - \norm{u(t)}\alpha(t)f(t)}{f(t)^2} \leq 0,\]
where the inequality follows from the assumption $\norm{u'(t)} \leq \norm{\alpha(t)u(t)}$. Thus $\frac{\norm{u(t)}}{f(t)}$ is decreasing, so that 
\[\frac{\norm{u(t)}}{f(t)} \leq \frac{\norm{u(a)}}{f(a)} = \norm{u(a)},\]
which is the desired inequality.
\end{proof}

\begin{theorem}
(ODE Uniqueness). Let $F:\mathcal{U}\to\mathcal{U}$ be Lipschitz and consider the differential equaton $u'(t)=F(u(t))$ with initial condition $u(a) = u_a$ for some function $u:I\to\mathcal{U}$, where $I = [a,b]$, or $[a,b)$, or $[a,\infty)$. Solutions to this equation are unique.
\end{theorem}
\label{thm:gronwalluniqueness}
\begin{proof}
Suppose there are two solutions $u(t)$ and $v(t)$, and let $g(t) = \norm{u(t)-v(t)}^2$. By assumption, there exists a constant $L_F$ such that $\norm{F(u(t))-F(v(t))} \leq L_F \norm{u(t)-v(t)}$, so that
\[\begin{aligned}
g'(t) &= 2\norm{u(t)-v(t)}\norm{u'(t)-v'(t)}\\
&=2\norm{u(t)-v(t)}\norm{F(u(t))-F(v(t))}\\
&\leq 2L_F\norm{u(t)-v(t)}^2\\
&= 2L_F g(t).
\end{aligned}\]
Notice that $\alpha := 2L_F$ is a positive continuous function, so we may apply Gr\"{o}nwall's inequality to $g(t)$ to conclude
\[g(t) \leq g(a)e^{2L_f(t-a)} = 0,\]
since $g(a)=0$.
\end{proof}

\section{Note on Generators of Parallel Transport}
Consider the differential equation $\dot{\rho}(\mu) = i[K_B,\rho(\mu)]$ with initial condition $\rho(0)=P_B(0)$. Here $K_B = \int_\mathbb{R}W_\gamma(t)e^{-itH_B}\dot{H_B}e^{itH_B}dt$, and recall that in our setting, $\dot{H_B} = Q_h$. We know that the solution is $\rho(\mu) = P_B(\mu)$ (proposition \ref{prop:generatorparalleltransport}). Notice that the map $F:\mathcal{U}\to\mathcal{U}$ defined by $F(A) = i[K_B,A]$ is Lipschitz, since
\[\norm{F(A)-F(B)} = \norm{[K_B,A-B]} \leq 2\norm{K_B}\norm{A-B}.\]
The Lipschitz constant is $2\norm{K_B}$, which is finite since $K_B$ is a bounded operator:

\[\begin{aligned}
\norm{K_B} &\leq \int_\mathbb{R}|W_\gamma(t)| \norm{e^{-itH_B}Q_he^{itH_B}}dt\leq \int_\mathbb{R}|W_\gamma(t)|dt\norm{Q_h}.
\end{aligned}\]

Indeed, since $Q_h$ is the number operator on a finite volume, by charge conservation and the fact that the dimension of the Hilbert space is bounded uniformly by $d$, there can only be a finite number of charges in the region $\Gamma_h$.

Thus, by Gr\"{o}nwall's uniqueness theorem (appendix \ref{thm:gronwalluniqueness}), we see that the solution to the equation $\dot{\rho}(\mu) = F(\rho(\mu)) = i[K_B,\rho(\mu)]$ is unique. 

Now define 

\[K_E := \int_\mathbb{R} W_\gamma(t) e^{-itH_E}Q_he^{itH_E}dt,\]

which is using the gap $\gamma$ of $H_B$ to define $W_\gamma$, but also using the edge Hamiltonian in the time evolution operators. Consider $\sigma:[0,\infty)\to\mathcal{U}$ defined by

\[\dot{\sigma}(\mu) = i[K_E, \sigma(\mu)] \quad\quad\quad\quad \sigma(0) = P_E(0).\]

We now show that, similar to how $\rho$ is an approximation of $P_B$, $\sigma$ is also a good approximation of $P_E$ (up to $\mathcal{O}(L^{-\infty})$) ``in the bulk". Let $A\in \Gamma_B$ be an operator localized in the bulk of the edge system. Then 

\[\begin{aligned}
\Tr(\dot{\sigma} A) &= \Tr (i[K_E, \sigma]A)\\
&= \Tr(i[A,K_E]\sigma)\\
&= \int_\mathbb{R}W_\gamma(t) \Tr([e^{-itH_E}Q_he^{itH_E}, A] \sigma)dt\\
&= \int_\mathbb{R}W_\gamma(t) \Tr(e^{-itH_E}[Q_h, e^{itH_E}Ae^{-itH_E}]e^{itH_E} \sigma)dt\\
&= \int_\mathbb{R}W_\gamma(t) \Tr(e^{-itH_E}[Q_h, e^{itH_B}Ae^{-itH_B}]e^{itH_E} + \mathcal{O}(L^{-\infty})\sigma)dt\\
&= \int_\mathbb{R}W_\gamma(t) \Tr(e^{-itH_B}[Q_h, e^{itH_B}Ae^{-itH_B}]e^{itH_B} + \mathcal{O}(L^{-\infty})\sigma)dt\\
&= \int_\mathbb{R}W_\gamma(t) \Tr([e^{-itH_B}Q_he^{itH_B}, A] \sigma)dt + \mathcal{O}(L^{-\infty})\\
&= \Tr(i[A,K_B]\sigma] + \mathcal{O}(L^{-\infty})\\
&= \Tr(i[K_B,\sigma]A) + \mathcal{O}(L^{-\infty}),
\end{aligned}\]

since $\sigma$ is trace-class (?) and $W_\gamma \in L^1$. By linearity of the trace, we see that $\Tr((\dot{\sigma} - i[K_B,\sigma])A) = \mathcal{O}(L^{-\infty})$ for any operator $A \in \Gamma_B$ (does this mean $\dot{\sigma}-i[K_E,\sigma] = 0$?). But the solution of $\dot{\sigma} - i[K_B,\sigma] = 0$ (with initial condition $\sigma(0)=P_B(0)$) is unique; it is $\rho(\mu)$, or $P_B(\mu)$. Hence

\[\Tr(P_E A) = \Tr(P_B A) + \mathcal{O}(L^{-\infty}) = \Tr(\rho A) + \mathcal{O}(L^{-\infty}) = Tr(\sigma A) + \mathcal{O}(L^{-\infty}) \]

for any operator $A\in\Gamma_B$. In particular, this gives another local formula for the Hall conductivity in the bulk of an edge system, by taking $A = J_V$, where $J$ is the current operator and $V \subset \Gamma_B$ is a set localized in the bulk. The Hall conductivity is given by $\Tr(\dot{P_E}J_V)$, and this can be approximated by

\[\Tr(\dot{P_E} J_V) = \Tr(\dot{P_B} J_V) + \mathcal{O}(L^{-\infty}) = \Tr(\dot{\rho} J_V) + \mathcal{O}(L^{-\infty}) = \Tr(\dot{\sigma} J_V) + \mathcal{O}(L^{-\infty}).\]

Want to pick a norm s.t. Gronwall gives $\norm{\rho(\mu)-\sigma(\mu)}_G \leq \norm{P_B(0)-P_E(0)}_Ge^{2L_F\mu}$. Need $\norm{P_B(0)-P_E(0)}_G$ to be small enough to kill the exponential which depends on $L_F = 2\norm{K_B}_G \leq \norm{W_\gamma}_{L^1}\norm{Q_h}_G$. If we use the operator norm for $\norm{\cdot}_G$, we would get $\norm{Q_h}_G = d|\Gamma_h|$ in the exponent. Need $\norm{\cdot}_G$ to be an actual norm so that $\norm{\rho-\sigma}_G = 0 \implies \rho = \sigma$.

\subsection*{From Dec 13 Meeting}

Let $r(t) = \rho(t)-\sigma(t)$. Notice that 

\[\frac{d}{dt} e^{itK_B}\sigma_0e^{-itK_B} = e^{itK_B}i[K_B,\sigma_0]e^{-itK_B} + e^{-itK_B}\dot{\sigma_0}e^{itK_B}.\]










\section{A Section}
Here is a section with some text.  Equations look like this
$y=x$.\footnote{Here is a footnote.}

This is an example of a second paragraph in a section so you can
see how much it is indented by.

% Subsections follow
\subsection{This is a Subsection}
Here is an example of a citation: \cite{Forbes:2006ba}.  The actual
form of the citation is governed by the bibliographystyle.  These
citations are maintained in a BIBTeX file \texttt{sample.bib}.  You
could type these directly into the file.  For an example of the format
to use look at the file \texttt{ubcsample.bbl} after you compile this
file.\footnote{Here is another footnote.}

This is an example of a second paragraph in a subsection so you can
see how much it is indented by.

\subsubsection{This is a Subsubsection}
Here are some more citations \cite{LL3:1977,Peccei:1989,Turner:1999}.
If you use the \texttt{natbib} package with the \verb+sort&compress+
option, then the following citation will look the same as the first
citation in this section: \cite{Turner:1999,Peccei:1989,LL3:1977}.

This is an example of a second paragraph in a subsubsection so you can
see how much it is indented by.

\paragraph{This is a Paragraph}
Paragraphs and subparagraphs are the smallest units of text.  There is
no subsubsubsection etc.

\subparagraph{This is a Subparagraph}
This is the last level of organisation.  If you need more than this,
you should consider reorganizing your work\dots

\begin{equation}
  \mathrm{f}(x)=\int_{-\infty}^{\int_{-\infty}^x
    e^{-\frac{y^2}{2}}\mathrm{d}{y}}e^{-z^2}\mathrm{d}z
\end{equation}

In order to show you what a separate page would look like (i.e. without
a chapter heading) I must type some more text.  Thus I will babble a
bit and keep babbling for at least one more page\ldots  What you
should notice is that the chapter titles appear substantially lower
than the continuing text. Babble babble
babble babble babble babble babble babble babble babble babble babble
babble babble babble babble babble babble babble babble babble babble
babble babble babble babble babble babble babble babble babble babble
babble babble babble babble babble babble babble babble babble.

Babble babble babble babble babble babble babble babble babble babble
babble babble babble babble babble babble babble babble babble babble
babble babble babble babble babble babble babble babble babble babble
babble babble babble babble babble babble babble babble babble babble
babble babble babble babble babble babble babble babble babble babble
babble babble babble babble babble babble babble babble babble babble
babble babble babble babble babble babble babble babble babble babble
babble babble babble babble babble babble babble babble babble babble
babble babble babble babble babble babble babble babble babble babble
babble babble babble babble babble babble babble babble babble babble
babble babble babble babble babble babble babble babble babble babble
babble babble babble babble babble babble babble babble babble babble
babble babble babble babble.

\begin{table}[t]                 % optional [t, b or h];
  \begin{tabular}{|r||r@{.}l|}
    \hline
    Phoenix & \$960&35\\
    \hline
    Calgary & \$250&00\\
    \hline
  \end{tabular}
  \caption[Here is the caption for this wonderful table\ldots]{
    \label{tab:Table1}
    Here is the caption for this wonderful table. It has not been
    centered and the positioning has been specified to be at the top
    of the page.  Thus it appears above the babble rather than below
    where it is defined in the source file.}
\end{table}

% Force a new page: without this, the quote would appear on the
% previous page.
\newpage

\section{Quote}
Here is a quote:
\begin{quote}
  % It is centered
  \begin{center}
    This is a small poem,\\
    a little poem, a Haiku,\\
    to show you how to.\\
    ---Michael M$^{\rm c}$Neil Forbes.
  \end{center}
\end{quote}

This small poem shows several features:
\begin{itemize}
\item The use of the \verb|quote| and \verb|center| environments.
\item The \verb|\newpage| command has been used to force a page
  break.  (Sections do not usually start on a new page.)
\item The pagestyle has been set to suppress the headers using the
  command \verb|\thispagestyle{plain}|.  Note that using
  \verb|\pagestyle{plain}| would have affected all of the subsequent
  pages.
\end{itemize}
\section{Programs}
Here we give an example of a new float as defined using the
\texttt{float} package.  In the preamble we have used the commands
\begin{verbatim}
\floatstyle{ruled}
\newfloat{Program}{htbp}{lop}[chapter]
\end{verbatim}
This creates a ``Program'' environment that may be used for program
fragments.  A sample \texttt{python} program is shown in
Program~\ref{prog:fib}.  (Note that Python places a fairly restrictive
limit on recursion so trying to call this with a large $n$ before
building up the cache is likely to fail unless you increase the
recursion depth.)
\begin{Program}
  \caption{\label{prog:fib} Python program that computes the $n^{\rm
      th}$ Fibonacci number using memoization.}
\begin{verbatim}
def fib(n,_cache={}):
    if n < 2:
        return 1
    if n in _cache:
        return _cache[n]
    else:
        result = fib(n-1)+fib(n-2)
        _cache[n] = result
        return result
\end{verbatim}
\end{Program}
Instead of using a \texttt{verbatim} environment for your program
chunks, you might like to \texttt{include} them within an
\texttt{alltt} envrironment by including the \verb|\usepackage{alltt}|
package (see page 187 of the \LaTeX{} book).  Another useful package
is the \verb|\usepackage{listings}| which can pretty-print many
different types of source code.

% Force a new page
\newpage

%% Here we provide a short optional argument to \chapter[]{}.  This
%% optional argument will appear in the table of contents.  For long
%% titles, one should use this to give a single-line entry to the
%% table of contents.
\chapter[Another Chapter\ldots]{Another Chapter with a Very Long
  Chapter-name that will Probably Cause Problems}
\label{cha:apple_ref}

This chapter name is very long and does not display properly in the
running headers or in the table of contents.  To deal with this, we
provide a shorter version of the title as the optional argument to the
\verb|\chapter[]{}| command.

For example, this chapter's title and associated table of contents heading and
running header was created with\\
\verb|\chapter[Another Chapter\ldots]{Another Chapter with a Very Long|\\
\verb|Chapter-name that will Probably Cause Problems}|.

Note that, according to the thesis regulations, the heading included
in the table of contents must be a truncation of the actual heading.

This Chapter was used as a demonstration in the Preface for how to
attribute contribution from collaborators.  If there are any such
contributions, details must be included in the Preface.  If you wish,
you may additionally use a footnote such as this.\footnote{This
  chapter is based on work conducted in UBC's Maple Syrup Laboratory
  by Dr. A. Apple, Professor B. Boat, and C. Cat.}

\section{Another Section}
Another bunch of text to demonstrate what this file does.
You might want a list for example:\footnote{Here is a footnote in a
  different chapter.  Footnotes should come after punctuation.}
\begin{itemize}
\item An item in a list.
\item Another item in a list.
\end{itemize}

\section*{An Unnumbered Section That is Not Included in the Table of
  Contents}
\begin{figure}[ht]
  \begin{center}
%% psfrag: comment the following line if not using the psfrag package
    \psfrag{pie makes me happy!}{$\pi$ makes me happy!}
%% includegraphics: comment the following if not using the graphicx package
    \includegraphics[width=0.4\textwidth]{fig-eps-converted-to.pdf}
    \caption[Happy Face: figure example.]{\label{fig:happy} This is a figure of
      a happy face with a \texttt{psfrag} replacement.  The original figure
      (drawn in xfig and exported to a .eps file) has the text ``pie makes me
      happy!''.  The \texttt{psfrag} package replaces this with ``$\pi$ makes me
      happy!''.  Note: the Makefile compiles the sample using pdf\LaTeX\ which
      cannot use \texttt{psfrag} directly.  For some options that work with
      pdf\LaTeX, please see this discussion:
      \url{http://tex.stackexchange.com/questions/11839}.  For the caption, we
      have used the optional argument for the caption command so that only a
      short version of this caption occurs in the list of figures.}
  \end{center}
\end{figure}
\afterpage{\clearpage}
Here is an example of a figure environment.
Perhaps I should say that the example of a figure can be seen in
Figure~\ref{fig:happy}.  Figure placement can be tricky with \LaTeX\
because figures and tables are treated as ``floats'': text can flow
around them, but if there is not enough space, they will appear later.
To prevent figures from going too far, the
\verb|\afterpage{\clearpage}| command can be used.  This makes sure
that the figure are typeset at the end of the page (possibly appear on
their own on the following pages) and before any subsequent text.

The \verb|\clearpage| forces a page break so that the figure can be
placed, but without the the \verb|\afterpage{}| command, the page
would be broken too early (at the \verb|\clearpage| statement).  The
\verb|\afterpage{}| command tells \LaTeX{} to issue the command after
the present page has been rendered.

\section{Tables}
We have already included one table:~\ref{tab:Table1}.  Another table
is plopped right here.
\begin{table}[ht]
  \begin{center}
    \begin{tabular}{|l||l|l||l|l|}
      \hline
      &\multicolumn{2}{l|}{Singular}&\multicolumn{2}{l|}{Plural}\\
      \cline{2-5}
       &English&\textbf{Gaeilge}&English&\textbf{Gaeilge}\\
      \hline\hline
      1st Person&at me&\textbf{agam}&at us&\textbf{againn}\\
      2nd Person&at you&\textbf{agat}&at you&\textbf{agaibh}\\
      3rd Person&at him&\textbf{aige}&at them&\textbf{acu}\\
       &at her&\textbf{aici}& & \\
      \hline
    \end{tabular}
    \caption{
      \label{tab:Table2}
      Another table.}
  \end{center}
\end{table}
Well, actually, as with Figures, tables do not
necessarily appear right ``here'' because tables are also ``floats''.
\LaTeX{} puts them where it can.  Because of this, one should refer to
floats by their labels rather than by their location.  This example is
demonstrated by Table~\ref{tab:Table2}.  This one is pretty close,
however.  (Note: you should generally not put tables or figures in the
middle of a paragraph.  This example is for demonstration purposes
only.)

Another useful package is \verb|\usepackage{longtable}| which provides
the \texttt{longtable} environment.  This is nice because it allows
tables to span multiple pages.  Table~\ref{tab:longtable} has been
formatted this way.
\begin{center}
  \begin{longtable}{|l|l|l|}
    \caption{\label{tab:longtable}Feasible triples for
      highly variable Grid}\\

    \hline \multicolumn{1}{|c|}{\textbf{Time (s)}} &
    \multicolumn{1}{c|}{\textbf{Triple chosen}} &
    \multicolumn{1}{c|}{\textbf{Other feasible triples}} \\ \hline
    \endfirsthead

    \multicolumn{3}{c}%
    {{\bfseries \tablename\ \thetable{} -- continued from previous page}} \\
    \hline \multicolumn{1}{|c|}{\textbf{Time (s)}} &
    \multicolumn{1}{c|}{\textbf{Triple chosen}} &
    \multicolumn{1}{c|}{\textbf{Other feasible triples}} \\ \hline
    \endhead

    \hline \multicolumn{3}{|r|}{{Continued on next page}} \\ \hline
    \endfoot

    \hline \hline
    \endlastfoot

    0 & (1, 11, 13725) & (1, 12, 10980), (1, 13, 8235), (2, 2, 0), (3, 1, 0) \\
    274 & (1, 12, 10980) & (1, 13, 8235), (2, 2, 0), (2, 3, 0), (3, 1, 0) \\
    5490 & (1, 12, 13725) & (2, 2, 2745), (2, 3, 0), (3, 1, 0) \\
    8235 & (1, 12, 16470) & (1, 13, 13725), (2, 2, 2745), (2, 3, 0), (3, 1, 0) \\
    10980 & (1, 12, 16470) & (1, 13, 13725), (2, 2, 2745), (2, 3, 0), (3, 1, 0) \\
    13725 & (1, 12, 16470) & (1, 13, 13725), (2, 2, 2745), (2, 3, 0), (3, 1, 0) \\
    16470 & (1, 13, 16470) & (2, 2, 2745), (2, 3, 0), (3, 1, 0) \\
    19215 & (1, 12, 16470) & (1, 13, 13725), (2, 2, 2745), (2, 3, 0), (3, 1, 0) \\
    21960 & (1, 12, 16470) & (1, 13, 13725), (2, 2, 2745), (2, 3, 0), (3, 1, 0) \\
    24705 & (1, 12, 16470) & (1, 13, 13725), (2, 2, 2745), (2, 3, 0), (3, 1, 0) \\
    27450 & (1, 12, 16470) & (1, 13, 13725), (2, 2, 2745), (2, 3, 0), (3, 1, 0) \\
    30195 & (2, 2, 2745) & (2, 3, 0), (3, 1, 0) \\
    32940 & (1, 13, 16470) & (2, 2, 2745), (2, 3, 0), (3, 1, 0) \\
    35685 & (1, 13, 13725) & (2, 2, 2745), (2, 3, 0), (3, 1, 0) \\
    38430 & (1, 13, 10980) & (2, 2, 2745), (2, 3, 0), (3, 1, 0) \\
    41175 & (1, 12, 13725) & (1, 13, 10980), (2, 2, 2745), (2, 3, 0), (3, 1, 0) \\
    43920 & (1, 13, 10980) & (2, 2, 2745), (2, 3, 0), (3, 1, 0) \\
    46665 & (2, 2, 2745) & (2, 3, 0), (3, 1, 0) \\
    49410 & (2, 2, 2745) & (2, 3, 0), (3, 1, 0) \\
    52155 & (1, 12, 16470) & (1, 13, 13725), (2, 2, 2745), (2, 3, 0), (3, 1, 0) \\
    54900 & (1, 13, 13725) & (2, 2, 2745), (2, 3, 0), (3, 1, 0) \\
    57645 & (1, 13, 13725) & (2, 2, 2745), (2, 3, 0), (3, 1, 0) \\
    60390 & (1, 12, 13725) & (2, 2, 2745), (2, 3, 0), (3, 1, 0) \\
    63135 & (1, 13, 16470) & (2, 2, 2745), (2, 3, 0), (3, 1, 0) \\
    65880 & (1, 13, 16470) & (2, 2, 2745), (2, 3, 0), (3, 1, 0) \\
    68625 & (2, 2, 2745) & (2, 3, 0), (3, 1, 0) \\
    71370 & (1, 13, 13725) & (2, 2, 2745), (2, 3, 0), (3, 1, 0) \\
    74115 & (1, 12, 13725) & (2, 2, 2745), (2, 3, 0), (3, 1, 0) \\
    76860 & (1, 13, 13725) & (2, 2, 2745), (2, 3, 0), (3, 1, 0) \\
    79605 & (1, 13, 13725) & (2, 2, 2745), (2, 3, 0), (3, 1, 0) \\
    82350 & (1, 12, 13725) & (2, 2, 2745), (2, 3, 0), (3, 1, 0) \\
    85095 & (1, 12, 13725) & (1, 13, 10980), (2, 2, 2745), (2, 3, 0), (3, 1, 0) \\
    87840 & (1, 13, 16470) & (2, 2, 2745), (2, 3, 0), (3, 1, 0) \\
    90585 & (1, 13, 16470) & (2, 2, 2745), (2, 3, 0), (3, 1, 0) \\
    93330 & (1, 13, 13725) & (2, 2, 2745), (2, 3, 0), (3, 1, 0) \\
    96075 & (1, 13, 16470) & (2, 2, 2745), (2, 3, 0), (3, 1, 0) \\
    98820 & (1, 13, 16470) & (2, 2, 2745), (2, 3, 0), (3, 1, 0) \\
    101565 & (1, 13, 13725) & (2, 2, 2745), (2, 3, 0), (3, 1, 0) \\
    104310 & (1, 13, 16470) & (2, 2, 2745), (2, 3, 0), (3, 1, 0) \\
    107055 & (1, 13, 13725) & (2, 2, 2745), (2, 3, 0), (3, 1, 0) \\
    109800 & (1, 13, 13725) & (2, 2, 2745), (2, 3, 0), (3, 1, 0) \\
    112545 & (1, 12, 16470) & (1, 13, 13725), (2, 2, 2745), (2, 3, 0), (3, 1, 0) \\
    115290 & (1, 13, 16470) & (2, 2, 2745), (2, 3, 0), (3, 1, 0) \\
    118035 & (1, 13, 13725) & (2, 2, 2745), (2, 3, 0), (3, 1, 0) \\
    120780 & (1, 13, 16470) & (2, 2, 2745), (2, 3, 0), (3, 1, 0) \\
    123525 & (1, 13, 13725) & (2, 2, 2745), (2, 3, 0), (3, 1, 0) \\
    126270 & (1, 12, 16470) & (1, 13, 13725), (2, 2, 2745), (2, 3, 0), (3, 1, 0) \\
    129015 & (2, 2, 2745) & (2, 3, 0), (3, 1, 0) \\
    131760 & (2, 2, 2745) & (2, 3, 0), (3, 1, 0) \\
    134505 & (1, 13, 16470) & (2, 2, 2745), (2, 3, 0), (3, 1, 0) \\
    137250 & (1, 13, 13725) & (2, 2, 2745), (2, 3, 0), (3, 1, 0) \\
    139995 & (2, 2, 2745) & (2, 3, 0), (3, 1, 0) \\
    142740 & (2, 2, 2745) & (2, 3, 0), (3, 1, 0) \\
    145485 & (1, 12, 16470) & (1, 13, 13725), (2, 2, 2745), (2, 3, 0), (3, 1, 0) \\
    148230 & (2, 2, 2745) & (2, 3, 0), (3, 1, 0) \\
    150975 & (1, 13, 16470) & (2, 2, 2745), (2, 3, 0), (3, 1, 0) \\
    153720 & (1, 12, 13725) & (2, 2, 2745), (2, 3, 0), (3, 1, 0) \\
    156465 & (1, 13, 13725) & (2, 2, 2745), (2, 3, 0), (3, 1, 0) \\
    159210 & (1, 13, 13725) & (2, 2, 2745), (2, 3, 0), (3, 1, 0) \\
    161955 & (1, 13, 16470) & (2, 2, 2745), (2, 3, 0), (3, 1, 0) \\
    164700 & (1, 13, 13725) & (2, 2, 2745), (2, 3, 0), (3, 1, 0) \\
\end{longtable}
\end{center}

\subsection*{An Unnumbered Subsection}
Note that if you use subsections or further divisions under an
unnumbered section, then you should make them unnumbered as well
otherwise you will end up with zeros in the section numbering.

\chapter{Landscape Mode}
The landscape mode allows you to rotate a page through 90 degrees.  It
is generally not a good idea to make the chapter heading landscape,
but it can be useful for long tables etc.

\begin{landscape}
  This text should appear rotated, allowing for formatting of very
  wide tables etc.  Note that this might only work after you convert
  the \texttt{dvi} file to a postscript (\texttt{ps}) or \texttt{pdf}
  file using \texttt{dvips} or \texttt{dvipdf} etc.  This feature is
  provided by the \verb|lscape| and the \verb|pdflscape| packages.
  The latter is preferred if it works as it also rotates the pages in
  the pdf file for easier viewing.
\end{landscape}

%% This file is setup to use a bibtex file sample.bib and uses the
%% plain style.  Other styles may be used depending on the conventions
%% of your field of study.
%%
%%% Note: the bibliography must come before the appendices.
\bibliographystyle{plain}
\bibliography{sample}

%% Use this to reset the appendix counter.  Note that the FoGS
%% requires that the word ``Appendices'' appear in the table of
%% contents either before each appendix lable or as a division
%% denoting the start of the appendices.  We take the latter option
%% here.  This is ensured by making the \texttt{appendicestoc} option
%% a default option to the UBC thesis class.

%%% If you only have one appendix, please uncomment the following line.
% \renewcommand{\appendicesname}{Appendix}
\appendix
\chapter{First Appendix}
Here you can have your appendices.  Note that if you only have a
single appendix, you should issue
\verb|\renewcommand{\appendicesname}{Appendix}| before calling
\verb|\appendix| to display the singular ``Appendix'' rather than the
default plural ``Appendices''.

\chapter{Second Appendix}
Here is the second appendix.

%% This changes the headings and chapter titles (no numbers for
%% example).
\backmatter

%% Indices come here if you have them.

\chapter*{Additional Information}
This chapter shows you how to include additional information in your
thesis, the removal of which will not affect the submission.  Such
material should be removed before the thesis is actually submitted.

First, the chapter is unnumbered and not included in the Table of
Contents.  Second, it is the last section of the thesis, so its
removal will not alter any of the page numbering etc. for the previous
sections.  Do not include any floats, however, as these will appear in
the initial lists.

The \texttt{ubcthesis} \LaTeX{} class has been designed to aid you in
producing a thesis that conforms to the requirements of The
University of British Columbia Faculty of Graduate Studies (FoGS).

Proper use of this class and sample is highly recommended---and should
produce a well formatted document that meets the FoGS requirement.
Notwithstanding, complex theses may require additional formatting that
may conflict with some of the requirements.  We therefore \emph{highly
  recommend} that you consult one of the FoGS staff for assistance and
an assessment of potential problems \emph{before} starting final
draft.

While we have attemped to address most of the thesis formatting
requirements in these files, they do not constitute an official set of
thesis requirements.  The official requirements are available at the
following section of the FoGS web site:
\begin{center}
  \begin{tabular}{|l|}
    \hline
    \url{http://www.grad.ubc.ca/current-students/dissertation-thesis-preparation}\\
    \hline
  \end{tabular}
\end{center}
We recommend that you review these instructions carefully.

\end{document}
\endinput
%%
%% End of file `ubcsample.tex'.
